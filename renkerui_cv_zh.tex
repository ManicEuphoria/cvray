\documentclass[11pt, a4paper,sans]{moderncv}   % possible options include font size ('10pt', '11pt' and '12pt'), paper size ('a4paper', 'letterpaper', 'a5paper', 'legalpaper', 'executivepaper' and 'landscape') and font family ('sans' and 'roman')
\usepackage{CTEX}

\moderncvstyle{banking}
\moderncvcolor{purple}
\nopagenumbers{}

% if you are not using xelatex ou lualatex, replace by the encoding you are using
\usepackage[utf8]{inputenc}

\usepackage{geometry}
\geometry{left=1.7cm,right=1.8cm,top=1cm,bottom=1cm}

\usepackage{import}

\name{\yahei 任科叡}{}
\title{\yahei \LARGE Ruby开发工程师}

\phone[mobile]{+86 15908143240}
\email{weirdhoney.mizar@gmail.com}
\social[github]{ManicEuphoria}

\begin{document}

\makecvtitle
\vspace{-38pt}

\section{\yahei 教育背景}
\begin{itemize}
  \item{\yahei 工学学士,软件工程,\textbf{武汉大学}\hfill 武汉,\textbf{09/2011 -- 07/2015}}
\end{itemize}

\vspace{-8pt}

\section{\yahei 工作经历}

\begin{itemize}

  \item{\yahei \textbf{Big Bang Games,Ruby服务器端工程师} \hfill 成都,\textbf{10/2018 -- 05/2019}}
  \begin{itemize}\yahei
    \item 负责Panda Slots服务器端的开发工作,Panda Slots是一款赌场类模拟手游,项目致力于开发出比肩业内大厂的赌场游戏。
    \item 项目基于Ruby on Rails,数据库基于PostgreSQL,使用RSpec作为测试框架,项目后期开发基于Docker环境。
    \item 主要负责对游戏新特性的需求分析,在编写测试为前提下驱动对应功能模块的开发,后续在小组内进行统一的code review并给予反馈完善并提交代码,工作期间经历了从小功能的开发到整个大模块的设计,中间遇到棘手的问题诸如修改已存在的数据库字段名导致关联模型失效等问题,通过深刻理解ActiveRecord与数据库之间关系后得以解决。
    \item 在职期间我开发出了赛事竞猜游戏、机台分级功能、房间多人游戏功能、每日转轮游戏和每日领取免费金币等功能,学会了以TDD为驱动进行开发,理解了如何使用OOD编写及重构代码。
  \end{itemize}

  \vspace{6pt}

  \item{\yahei \textbf{极验验证,Python后端工程师} \hfill 武汉,\textbf{12/2015 -- 07/2018}}

    \begin{itemize}\yahei
      \item 负责极验验证Web端面向用户及内部员工的管理平台的后端开发、集成与部署工作。
      \item 工作期间的三个项目分别基于Django、Tornado与Flask框架,数据库基于MongoDB及MySQL,以Redis作为缓存方案。
      \item 前期工作负责面向内部平台的验证数据与用户的管理功能开发;中期负责面向用户平台的重构、模块开发及部署工作;后期负责新项目框架搭建、任务分配、合作开发及帮助指导新人工作,期间对接平安UM账户体系是一个巨大的挑战,基于对方无远程可接入的公网的前提下在本地黑盒开发对接模块后出差对方总部现场入网调试。
      \item 在职期间经历了客户量从上万到25万的用户管理平台的重构与迭代,主导平安科技对接极验验证私有化管理后台的开发工作;为内部各部门提供了每天数以亿计验证数据的各维度分析报表及各种用户在运营、商务方面的数据分析功能。
    \end{itemize}

\end{itemize}
\vspace{-8pt}

\section{\yahei 技能树}
\begin{itemize}
  \item{\yahei \textbf{编程语言:} 熟悉Ruby、Python}
  \item{\yahei \textbf{语言框架:} 熟悉Ruby on Rails、Flask,了解Django、Tornado}
  \item{\yahei \textbf{操作系统:} 熟悉CentOS、Ubuntu、MacOS}
  \item{\yahei \textbf{数据库:} 了解PostgreSQL、MySQL、MongoDB、Redis}
  \item{\yahei \textbf{工具:} 掌握Git、Markdown、Vim,了解Docker}
  \item{\yahei \textbf{设计模式:} 了解RESTful、TDD、OOD}
  \item{\yahei \textbf{外语:} CET 4/6,良好的英文读写,无障碍看Youtube英语视频及满足日常交流的听说能力}
\end{itemize}
\vspace{-8pt}

\section{\yahei 自我评价}

\begin{itemize}
  \item{\yahei 在高效完成本职工作同时致力于写出整洁、可移植和可复用程序的代码洁癖者}
  \item{\yahei 一个以解决问题为驱动力,能快速学习新技能的人。日常解决问题主要依赖于Google、Github、Stack Overflow及各种API文档和技术资料, 同时热衷于各种英语技术论坛及博客}
  \item{\yahei 业余生活热衷于读书、吉他弹唱、跑步、篮球等。有较强的沟通能力及团队合作能力}
\end{itemize}

\end{document}
\documentclass[11pt, a4paper,sans]{moderncv}   % possible options include font size ('10pt', '11pt' and '12pt'), paper size ('a4paper', 'letterpaper', 'a5paper', 'legalpaper', 'executivepaper' and 'landscape') and font family ('sans' and 'roman')
\usepackage{CTEX}

\moderncvstyle{banking}
\moderncvcolor{purple}
\nopagenumbers{}

% if you are not using xelatex ou lualatex, replace by the encoding you are using
\usepackage[utf8]{inputenc}

\usepackage{geometry}
\geometry{left=1.7cm,right=1.8cm,top=1cm,bottom=1cm}

\usepackage{import}
\usepackage[T1]{fontenc}

\name{\yahei 任科叡}{}
\title{\yahei \LARGE Ruby开发工程师}

\phone[mobile]{+86 199-0888-9316}
\email{weirdhoney.mizar@gmail.com}
\social[github]{ManicEuphoria}

\begin{document}

\makecvtitle
\vspace{-38pt}

\section{\yahei 教育背景}
\begin{itemize}
  \item{\yahei 工学学士,软件工程,\textbf{武汉大学}\hfill 武汉,\textbf{09/2011 -- 07/2015}}
\end{itemize}

\vspace{-8pt}

\section{\yahei 工作经历}

\begin{itemize}
\item{\yahei \textbf{西瓜创客,Ruby后端工程师} \hfill 成都,\textbf{07/2019 -- 11/2019}}
\begin{itemize}\yahei
  \item 负责运营平台需求评审、技术评审以及相关研发、BUG值守工作。
  \item 在职时参与了多个基于Ruby的项目与微服务的研发与重构工作,有基于Ruby的微服务从0到1的经验,有k8s使用经验,高效的Git协作经验与CI/CD经验。
  \item 在职工作流程从需求或技术评审到预估时间节点并出具方案文档,开发过程中注重细节,以较高的代码质量提交PR给团队中的伙伴进行code review,后期配合QA部署k8s冒烟测试并上线。期间在设计促销活动微服务中能迅速根据产品的设计反复进行有效沟通汲取想法和意见,调整技术方案并落地,接手代号增大益的临时项目时能顶住业务方压力迅速出具有效的技术方案并一次通过评审。
  \item 在职期间对工作极度认真负责,具有高度的ownership,个人发起了多场复杂的技术评审并落地文档或代码,针对棘手的线上BUG锻炼出了与BUG提出的业务方进行有效沟通迅速后排查定位问题并及时热修复的能力,按时按质完成个人工作后能帮助团队内小伙伴一起成长进步。
  \end{itemize}

  \vspace{6pt}
\item
  \item{\yahei \textbf{Big Bang Games,Ruby服务器端工程师} \hfill 成都,\textbf{10/2018 -- 05/2019}}
  \begin{itemize}\yahei
    \item 负责赌场类模拟手游Panda Slots服务器端的开发工作。
    \item 项目基于Rails、PostgreSQL,使用RSpec作为测试框架,项目后期开发基于Docker环境。
    \item 主要负责对游戏新特性的需求分析,在TDD为前提下驱动对应功能模块的开发,后续在小组内进行统一的code review并给予反馈完善并提交代码,工作期间经历了从小功能的开发到整个大模块的设计,遇到棘手的问题不拖泥带水,设定时间内力争自己解决,设定时间外也能求助团队协助。
    \item 在职期间开发出了赛事竞猜游戏、机台分级功能、房间多人游戏功能、每日转轮游戏和每日领取免费金币等功能,深刻理解了TDD与OOD在开发中的运用及其重要性。
  \end{itemize}

  \vspace{6pt}

  \item{\yahei \textbf{极验验证,Python后端工程师} \hfill 武汉,\textbf{12/2015 -- 07/2018}}

    \begin{itemize}\yahei
      \item 负责极验验证Web端面向用户及内部员工的管理平台的后端开发、集成与部署工作。
      \item 工作期间负责与经手的4个项目分别基于Django、Tornado与Flask框架,有项目从0到1的经验。
      \item 前期工作负责面向内部平台的验证数据与用户的管理功能开发;中期负责面向用户平台的重构、模块开发及部署工作;后期负责新项目框架搭建、任务分配、合作开发及帮助指导新人工作,期间对接平安UM账户体系是一个巨大的挑战,基于对方无远程可接入的公网的前提下在本地黑盒开发对接模块后出差对方总部现场入网调试。
      \item 在职期间经历了客户量从上万到25万的用户管理平台的重构与迭代,主导平安科技对接极验验证私有化管理后台的开发工作;为内部各部门提供了每天验证数据的各维度分析报表及各种用户在运营、商务方面的数据分析功能。
    \end{itemize}

\end{itemize}
\vspace{-8pt}

\section{\yahei 技能树}
\begin{itemize}
  \item{\yahei \textbf{编程语言:} 掌握Ruby、Python}
  \item{\yahei \textbf{语言框架:} 掌握Ruby on Rails,熟悉部分Python Web开源框架}
  \item{\yahei \textbf{操作系统:} 掌握CentOS、Ubuntu、MacOS}
  \item{\yahei \textbf{数据库:} 掌握MySQL、MongoDB、Redis}
  \item{\yahei \textbf{工具:} 掌握UML,了解docker、k8s及CI/CD}
  \item{\yahei \textbf{设计模式:} 掌握RESTful、TDD、OOD,了解微服务}
  \item{\yahei \textbf{外语:} CET 4/6,良好的英文读写及满足出国日常交流的听说能力}
\end{itemize}
\vspace{-8pt}

\section{\yahei 自我评价}

\begin{itemize}
  \item{\yahei 对代码有严格要求:在开发的过程里合理运用OOD,有TDD为先的编码意识,并且力求产出的代码的整洁性、可复用性及可迁移性}
  \item{\yahei 高度责任心与驱动能力:面对压力不气馁不妥协,能在肩扛重任时迅速产出方案或解决思路,经手的事情力求能得到团队高度认可}
  \item{\yahei 良好的学习能力:能够无障碍阅读英文技术书籍并转化为个人知识,能迅速理解业务方的思路并进行高效沟通,能迅速学会一门乐器为生活带来快乐}
  \item{\yahei 热爱生活:喜欢阅读,会点吉他,唱歌不错,也不惧社交,篮球游泳各种运动信手拈来}
\end{itemize}

\end{document}
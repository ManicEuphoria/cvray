% Copyright 2006-2013 Xavier Danaux (xdanaux@gmail.com).
%
% This work may be distributed and/or modified under the
% conditions of the LaTeX Project Public License version 1.3c,
% available at http://www.latex-project.org/lppl/.

\documentclass[11pt,a4paper,sans]{moderncv}        % possible options include font size ('10pt', '11pt' and '12pt'), paper size ('a4paper', 'letterpaper', 'a5paper', 'legalpaper', 'executivepaper' and 'landscape') and font family ('sans' and 'roman')
\usepackage{CTEX}

% modern themes
\moderncvstyle{banking} % style options are 'casual' (default), 'classic', 'oldstyle' and 'banking'
\moderncvcolor{purple} % color options 'blue' (default), 'orange', 'green', 'red', 'purple', 'grey' and 'black'
%\renewcommand{\familydefault}{\sfdefault} % to set the default font; use '\sfdefault' for the default sans serif font, '\rmdefault' for the default roman one, or any tex font name

%\nopagenumbers{} % uncomment to suppress automatic page numbering for CVs longer than one page

% character encoding
\usepackage[utf8]{inputenc} % if you are not using xelatex ou lualatex, replace by the encoding you are using

% adjust the page margins
\usepackage{geometry}
\geometry{left=1.7cm,right=1.8cm,top=1cm,bottom=1cm}
%\setlength{\hintscolumnwidth}{3cm} % if you want to change the width of the column with the dates
%\setlength{\makecvtitlenamewidth}{10cm} % for the 'classic' style, if you want to force the width allocated to your name and avoid line breaks. be careful though, the length is normally calculated to avoid any overlap with your personal info; use this at your own typographical risks...

\usepackage{import}

% personal data
\name{\yahei 宋卫涛}{}
\title{\yahei \LARGE C++工程师}
% \address{武汉市 武汉大学 电子信息学院, 430072}% the "postcode city" and and "country" arguments can be omitted or provided empty
\phone[mobile]{176-1271-8964}
% \phone[mobile]{wechat: taotsi42}
%\phone[fax]{+3~(456)~789~012}
\email{tmkf.guthrie@gmail.com}
\social[github]{taotsi}
%\extrainfo{}
%\photo[64pt][0.4pt]{picture.eps} % '64pt' is the height the picture must be resized to, 0.4pt is the thickness of the frame around it (put it to 0pt for no frame) and 'picture' is the name of the picture file
%\quote{gimme lollipop}

\begin{document}

\makecvtitle

\vspace{-30pt}

\section{\yahei 教育背景}

\begin{itemize}

	\item{\yahei{硕士 \textbf{\ 武汉大学\ } \small{电子与通信工程}}\hfill\textbf{2017--2019}}

	\item{\yahei{学士 \textbf{\ 武汉大学\ } \small{电子信息工程}}\hfill\textbf{2013--2017}}

\end{itemize}

\vspace{-6pt}

\section{\yahei 项目经历}

\begin{itemize}

	\item{\textbf{\yahei 简化障碍物建模}
	      \hfill{\href{https://github.com/taotsi/DroneWorld}
		      {\footnotesize\underline{\emph{github.com/taotsi/DroneWorld}}}
		      \textbf{\ \ 2018--2019}}}

	      \vspace{2pt}

	      {\small\yahei 以几何方法实现的精简环境建模,在精简存储量的同时,使得信息损失于无人机极少。(基于城市障碍物的天然竖直属性,对深度图提取列像素并将障碍物识别为“pillar”,然后对pillar进行横向聚类与滤波,去除小于无人机尺寸的冗余信息,最终得到精简的3D结构。)负责工程实现,包括仿真环境的搭建以及算法的实现、改进和优化。仿真环境基于Airsim API,以现代C++编码。仿真环境框架基于循环和更新模式。各模块皆组件化,包括图像提取、列像素处理、横向聚类、3D建模、数据融合与运动控制。框架设计为具有可扩展性,用于实验室后续项目。会议论文在投,二作。}

	      \vspace{3pt}

	\item{\textbf{\yahei 2.4GHz无损音频收发器}
	      \hfill{\href{https://github.com/taotsi/AirTone-SW}
		      {\footnotesize\underline{\emph{github.com/taotsi/AirTone-SW}}}
		      \textbf{\ \ 2015--2017}}}

	      \vspace{2pt}

	      {\small\yahei 用于电声乐器和音频设备的无损收发器。基于数字RF芯片CC8530(40kHz/16bit)设计了音频流的无损传输系统。搭建了完整的单片机系统,包括硬件上的方案设计、PCB输出和焊接,以及软件上的音频流的管理、存储的管理、交互界面和电源管理等。自最底层起构建了完整的驱动库与上层应用,有较高的可移植性。在保证较低成本的同时,达到了(当时)市面上数字音频收发器的最高性能。有相应专利。}

\end{itemize}

\vspace{-6pt}

\section{\yahei 技能树}

\begin{itemize}

	\item{\yahei \textbf{现代C++}(泛型,STL,并发); \textbf{C}(单片机驱动,BSP); Python}
	\item{\yahei \textbf{Linux}(Arch,Ubuntu); \textbf{Shell}; Git}
	\item{\yahei \textbf{Altium Designer}; PCB焊接(QFPN,0402)}
	\item{\yahei 了解基本的数据结构与算法; 了解基本的设计模式}
	\item{\yahei 了解ROS, UE4}

\end{itemize}

\vspace{-6pt}

\section{\yahei 活动经历}

\begin{itemize}

	\item{\yahei{\textbf{武汉大学古琴文化研究所}\ \ \ 协助江柏安教授创立}\hfill\textbf{2015--2016}}

	      %   \vspace{1pt}

	      %   {\small\yahei 协助知名音乐教授江柏安创办了古琴文化研究所,出任第1届社长。执行了早期的课程建设和日常事务管理。}

	      %   \vspace{3pt}

	\item{\yahei{\textbf{武大制躁音乐节系列}\ \ \ \ \ \ \ \ \ 早期发起人,至今5届}\hfill\textbf{2014--2015}}

	      %   \vspace{1pt}

	      %   {\small\yahei 带领青鸟吉他协会创办了“武大制躁”音乐节(第1届),至今已有4届。致力于为本地职业/半职业乐队提供演出交流机会。每届观众约300-500人,来自武汉各高校。在武汉摇滚圈内有一定名气。}

\end{itemize}

\vspace{-6pt}

\section{\yahei 其他}

\subsection{\yahei 自我评价}

{\yahei\small 快速学习,任务驱动,熟悉英文环境,熟悉Github;喜欢会动的东西;向往以先进技术实现具有前沿价值的产品;追求稳定高效的工程实现、规范易读的代码和现代的团队开发方法。业余时间爱好读书与弹琴。}

\subsection{\yahei 业余项目}

\begin{tabular}{ l l }

	\yahei\footnotesize{* 一个模拟程序,致力于实现钢铁、枪炮与病菌中的模型,\href{https://github.com/taotsi/DingRL}{\underline{\emph{repo}}}}
	& \ \ \ \ \ \ \ \ \ \ \ \ \
	\yahei\footnotesize{* 维护着一个VPN (V2Ray)} \\
	\yahei\footnotesize{* 基于一种棋类游戏的强化学习(Python; \href{https://github.com/taotsi/DingRL}{\underline{\emph{repo}}})}
	& \ \ \ \ \ \ \ \ \ \ \ \ \
	\yahei\footnotesize{* 复现过Conway的生命游戏(Python)} \\
	\yahei\footnotesize{* 给女朋友写过一个小型笔记网站(Flask)}
	& \ \ \ \ \ \ \ \ \ \ \ \ \
	\yahei\footnotesize{* 复现过几个模拟电路效果器(AD16)} \\
	\yahei\footnotesize{* 复现过Reynolds的Boids(UE4; \href{https://github.com/taotsi/UE4Swarm}{\underline{\emph{repo}}})}
	% & \ \ \ \ \ \ \ \ \ \ \ \ \ \ \
\end{tabular}

\end{document}

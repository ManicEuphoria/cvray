%% start of file `template.tex'.
%% Copyright 2006-2015 Xavier Danaux (xdanaux@gmail.com).
%
% This work may be distributed and/or modified under the
% conditions of the LaTeX Project Public License version 1.3c,
% available at http://www.latex-project.org/lppl/.


\documentclass[12pt, a4paper,sans]{moderncv}       % possible options include font size ('10pt', '11pt' and '12pt'), paper size ('a4paper', 'letterpaper', 'a5paper', 'legalpaper', 'executivepaper' and 'landscape') and font family ('sans' and 'roman')
\pagenumbering{gobble}
% moderncv themes
\moderncvstyle{banking}                             % style options are 'casual' (default), 'classic', 'banking', 'oldstyle' and 'fancy'
\moderncvcolor{purple}                               % color options 'black', 'blue' (default), 'burgundy', 'green', 'grey', 'orange', 'purple' and 'red'
%\renewcommand{\familydefault}{\sfdefault}         % to set the default font; use '\sfdefault' for the default sans serif font, '\rmdefault' for the default roman one, or any tex font name
%\nopagenumbers{}                                  % uncomment to suppress automatic page numbering for CVs longer than one page

% character encoding
%\usepackage[utf8]{inputenc}                       % if you are not using xelatex ou lualatex, replace by the encoding you are using
%\usepackage{CJKutf8}                              % if you need to use CJK to typeset your moderncv_ray in Chinese, Japanese or Korean

% adjust the page margins
%\usepackage[scale=0.75]{geometry}
\usepackage{geometry}
\geometry{left=1.7cm,right=1.8cm,top=1cm,bottom=1cm}
%\setlength{\hintscolumnwidth}{3cm}                % if you want to change the width of the column with the dates
%\setlength{\makecvtitlenamewidth}{10cm}           % for the 'classic' style, if you want to force the width allocated to your name and avoid line breaks. be careful though, the length is normally calculated to avoid any overlap with your personal info; use this at your own typographical risks...

% personal data
\name{Kerui}{Ren}
\title{Ruby Engineer}                               % optional, remove / comment the line if not wanted
%\address{street and number}{postcode city}{country}% optional, remove / comment the line if not wanted; the "postcode city" and "country" arguments can be omitted or provided empty
\phone[mobile]{+86 199-0888-9316}                   % optional, remove / comment the line if not wanted; the optional "type" of the phone can be "mobile" (default), "fixed" or "fax"
\email{weirdhoney.mizar@gmail.com}                               % optional, remove / comment the line if not wanted
\social[github]{ManicEuphoria}
%\homepage{www.johndoe.com}                         % optional, remove / comment the line if not wanted
%\social[linkedin]{john.doe}                        % optional, remove / comment the line if not wanted
%\social[twitter]{jdoe}                             % optional, remove / comment the line if not wanted
%\extrainfo{additional information}                 % optional, remove / comment the line if not wanted
%\photo[64pt][0.4pt]{picture}                       % optional, remove / comment the line if not wanted; '64pt' is the height the picture must be resized to, 0.4pt is the thickness of the frame around it (put it to 0pt for no frame) and 'picture' is the name of the picture file
%\quote{Some quote}                                 % optional, remove / comment the line if not wanted

% bibliography adjustements (only useful if you make citations in your moderncv_ray, or print a list of publications using BibTeX)
%   to show numerical labels in the bibliography (default is to show no labels)
\makeatletter\renewcommand*{\bibliographyitemlabel}{\@biblabel{\arabic{enumiv}}}\makeatother
%   to redefine the bibliography heading string ("Publications")
%\renewcommand{\refname}{Articles}

% bibliography with mutiple entries
%\usepackage{multibib}
%\newcites{book,misc}{{Books},{Others}}
%----------------------------------------------------------------------------------
%            content
%----------------------------------------------------------------------------------
\begin{document}
    %\begin{CJK*}{UTF8}{gbsn}                          % to typeset your moderncv_ray in Chinese using CJK
    %-----       moderncv_ray       ---------------------------------------------------------
\makecvtitle
\vspace{-28pt}

\section{EDUCATION BACKGROUND}
    \begin{itemize}
        \item{\textbf{Wuhan University}, {Wuhan, China} \hfill\textbf{09/2011 - 07/2015}}
    \end{itemize}

    \begin{itemize}
        \item{\textit{Bechalar in Engineering, International School of Software}}
    \end{itemize}

\vspace{0pt}


\section{WORKING EXPERIENCE}

    \begin{itemize}

        \item{\textbf{Big Bang Games Co.Ltd, Ruby Engineer}, \hfill{Chengdu, China} \hfill\textbf{10/2018 - 05/2019}}
%        \item{\textit{Rudy Engineer}}
        \vspace{2pt}
%        \linebreak
        \begin{itemize}
            \item {\medium Worked in backend department, mainly in charge of features development of new games}
            \vspace{2pt}
            \item {\medium Responsible for analyzing new features of games, and designing of database models and relationship between them, to order to perfect their details. Refactor of existing features is also included}
            \vspace{2pt}
            \item {\medium Designed what public interface actually is and defined its routes, then wrote test script based on TDD, finally perfected its details based on tests when involved in development circle}
            \vspace{2pt}
            \item {\medium Designed several games and involved in the refactoring work on previous project, which greatly simplified existing structure and code, and the features I developed significantly kept the existing users}
        \end{itemize}


        \vspace{8pt}


        \item{\textbf{GEETEST Co.Ltd, Python Engineer}, \hfill{Wuhan, China} \hfill\textbf{12/2015 - 07/2018}}
        \vspace{2pt}
        \begin{itemize}
            \item { \medium Responsible for server development for all web projects in company}
            \item { \medium Mainly in charge of building backend framework, databases designs and implementation of business logic}
            \item { \medium Built management platform for enterprise users and individual users respectively, which increased from \textit{50,000} to \textit{250,000} during my work period}
            \item { \medium Led a three-people team to build internal management platform to provide stuffs with captcha data and relevant support to help them do better corporate management}
        \end{itemize}


    \end{itemize}
\vspace{0pt}
\section{SKILL TREE}
    \begin{itemize}
        \item{\textbf{Computer Languages:} {Skilled in \textit{Ruby}, \textit{Python}}}%
        \item{\textbf{Framework:} {Familiar with \textit{Ruby on Rails}, \textit{Flask}, also know some about \textit{Django}, \textit{Tornado}}}%
        \item{\textbf{OS:} {Operating \textit{CentOS}, \textit{Ubuntu}, \textit{MacOS} skillfully}}%
        \item{\textbf{DB:} {Basic understanding of \textit{PostgreSQL}, \textit{MySQL}, \textit{MongoDB}, \textit{Redis}}}%
        \item{\textbf{Tools:} {Intermediate level of \textit{Git}, \textit{Markdown} and \textit{vim}, know some about \textit{Docker}}}%
        \item{\textbf{Design Patterns:}{Basic understanding of \textit{RESTful}, \textit{TDD} and \textit{OOD}}}%
        \item{\textbf{Language}: {Good English literacy, with oral ability that satisfy basic conversations} }
    \end{itemize}

\vspace{0pt}
\section{OTHER INFORMATION}

\subsection{Self Assessment}

    \begin{itemize}
        \item{\medium A person who strictly follows various coding standards, always trying to keep my code clean, migratable and reusable}
        \item{\medium A task-driven, fast learner. I solve problems mainly by \textit{Google}, \textit{Github} and \textit{StackOverflow}, also \textit{API / Guide} documentations are helpful as well}
        \item{\medium An independent music lover and a guitar player. Also a good basketball player. Strong communication skills and teamwork ability with Git flow}
    \end{itemize}


\end{document}


%% end of file `template.tex'.
